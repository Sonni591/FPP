% !TeX document-id = {acd94374-0a5a-4c39-a15c-8f2e56f0151a}
\documentclass[12pt,a4paper, listof=entryprefix, bibliography=totocnumbered,toc=listofnumbered,lof=listofnumbered]{scrartcl}

% WICHTIG!!!
% Pseudokommentar um pdflatex zu erlauben andere Programme zu nutzen z.B. gnuplot
% !TeX TXS-program:compile = txs:///pdflatex/[--shell-escape] 

\usepackage[ngerman]{babel}
\usepackage[utf8]{inputenc}
\usepackage{amsmath}
\usepackage{nccmath}
\usepackage{amsfonts}
\usepackage{amssymb}
\usepackage{graphicx}
\usepackage{fancyhdr}
\usepackage{tabularx}
\usepackage{geometry}
\usepackage{setspace}
\usepackage[right]{eurosym}
\usepackage[printonlyused]{acronym}
\usepackage{subfig}
\usepackage{floatflt}
\usepackage[usenames,dvipsnames]{color}
\usepackage{colortbl}
\usepackage{xcolor}
\usepackage{paralist}
\usepackage{array}
\usepackage{titlesec}
\usepackage{parskip}
\usepackage{picinpar}
\usepackage[pdfpagelabels=true]{hyperref}
\usepackage{listings}
\usepackage{csquotes}
\usepackage{url}
\usepackage{float}
\usepackage{pgfplots}
\usepackage{paralist}
\usepackage[nonumberlist, nogroupskip]{glossaries}

%-----------------------------------------------------------------------------------
% Bibilothek
%-----------------------------------------------------------------------------------
% Einbinden des BibLateX paketes mit Ausgabeeinstellungen
\usepackage[
style=alphabetic,          % Zitierstil
maxbibnames=50,            % alle Autorennamen anzeigen
maxcitenames=4,            % maximale Namen, die im Kürzel angezeigt werden
autocite=inline,           % regelt Aussehen für \autocite (inline=\parancite)
block=space,               % kleiner horizontaler Platz zwischen den Feldern
backref=true,              % Seiten anzeigen, auf denen die Referenz vorkommt
backrefstyle=three+,       % fasst Seiten zusammen, z.B. S. 2f, 6ff, 7-10
date=short,                % Datumsformat
backend = biber,           % Backnend für Aufbereitung
]{biblatex}

%Zusätzliche für Umbrüche für Kleinbuchstaben z.B. in URLs
\appto\UrlBreaks{\do\a\do\b\do\c\do\d\do\e\do\f\do\g\do\h\do\i\do\j
	\do\k\do\l\do\m\do\n\do\o\do\p\do\q\do\r\do\s\do\t\do\u\do\v\do\w
	\do\x\do\y\do\z}


\newcounter{verzeichnis}
\setcounter{verzeichnis}{1}

%Abstände der Einträge
\setlength{\bibitemsep}{1em}     % Abstand zwischen den Literaturangaben
\setlength{\bibhang}{2em}        % Einzug nach jeweils erster Zeile

% Kürzel soll vier Buchstaben der Autoren enthalten statt drei
\DeclareLabelalphaTemplate{
	\labelelement{
		\field[final]{shorthand}
		\field{label}
		\field[strwidth=4,strside=left,ifnames=1]{labelname}
		\field[strwidth=2,strside=left,ifnames=2]{labelname}
		\field[strwidth=1,strside=left]{labelname}
	}
	\labelelement{
		\field[strwidth=2,strside=right]{year}
	}
}

% Bibliothek der Quellen
\bibliography{bib}
\label{bib}

% --------------------------------------------------------------------------------
% Einstellung für Listings
% --------------------------------------------------------------------------------
\lstset{basicstyle=\footnotesize, captionpos=b, breaklines=true, showstringspaces=false, tabsize=2, frame=lines, numbers=left, numberstyle=\tiny, xleftmargin=2em, framexleftmargin=2em}
\makeatletter
\def\l@lstlisting#1#2{\@dottedtocline{1}{0em}{1em}{\hspace{1,5em} Lst. #1}{#2}}
\makeatother

% --------------------------------------------------------------------------------
% Seitenformate
% --------------------------------------------------------------------------------
%Seitenformat
\geometry{a4paper, top=27mm, left=30mm, right=20mm, bottom=32mm, headsep=12mm, footskip=12mm}

% --------------------------------------------------------------------------------
% Metainformationen
% --------------------------------------------------------------------------------
\hypersetup{unicode=false, pdftoolbar=true, pdfmenubar=true, pdffitwindow=false, pdfstartview={FitH},
	pdftitle={Vorlage},
	pdfauthor={Stefan Jung},
	pdfsubject={Abschlussarbeit},
	pdfcreator={\LaTeX\ with package \flqq hyperref\frqq},
	pdfproducer={pdfTeX \the\pdftexversion.\pdftexrevision},
	pdfkeywords={Vorlage},
	pdfnewwindow=true,
	colorlinks=true,linkcolor=black,citecolor=black,filecolor=magenta,urlcolor=black}
\pdfinfo{/CreationDate (D:20141024101000)}
%\pgfplotsset{compat=1.11}

%-----------------------------------------------------------------------------------
% Abkürzungen AKRONYME HIER ERGÄNZEN
%-----------------------------------------------------------------------------------
\glssetwidest{OTHR}% Längste Abkürzung für eine korrekte Einrückung

%\makenoidxglossaries %Leeres Verzeichnis erstellen

%Abkürzungen hinzufügen
\newacronym{LIP}{LIP}{Labor für Informationstechnik und Produktionslogistik}
\newacronym{OTH}{OTHR}{Ostbayerische Technische Hochschule Regensburg}
\newacronym{CSS}{CSS}{Cascading Style Sheets}
\newacronym{PLSP}{PLSP}{Proportional Lotsizing and Scheduling Problem}
\newacronym{XML}{XML}{Extensible Markup Language}
\newacronym{CSV}{CSV}{Comma-separated values}

\begin{document}
% --------------------------------------------------------------------------------
% Globale Formateinstellungen
% --------------------------------------------------------------------------------
\onehalfspacing
% Abstände Überschrift
\titlespacing{\section}{0pt}{42pt}{6pt}
\titlespacing{\subsection}{0pt}{12pt}{6pt}
\titlespacing{\subsubsection}{0pt}{12pt}{6pt}

% Kopf- und Fusszeile
\pagestyle{fancy}
\lhead{}\chead{}
\rhead{\thesection\space\contentsname}
\lhead{}\cfoot{}
\rfoot{\ \linebreak \thepage}
\renewcommand{\headrulewidth}{0.4pt}
\renewcommand{\footrulewidth}{0.4pt}

% Nummereriung
\renewcommand{\thesection}{\Roman{section}}
\renewcommand{\theHsection}{\Roman{section}}
\pagenumbering{Roman}

% eigene Farbdefinitionen
\definecolor{lip}{HTML}{3366FF}
\definecolor{grey}{HTML}{ABABAB}

% ---------------------------------------------------------------------------
% Titelseite
% ---------------------------------------------------------------------------
\thispagestyle{empty}

%LIP Schriftzug in eigener Farbe 
\textsf{\begin{minipage}{.69\textwidth}
	\large
	\textcolor{lip}{\textbf{Labor für Informationstechnik und\\Produktionslogistik (LIP)}} %Farbe setzten
	\small 
	\textbf{\\Verfahren, Strategien, Prozesse und IT-Systeme}
	\\Professor Dr.-Ing. Frank Herrmann
\end{minipage}
%Einbinden des OTH Logos mir rechtsbündiger Ausrichtung
\begin{minipage}{.29\textwidth}
	\begin{flushright}
		\includegraphics[scale=.15]{Bilder/othlogo}\\
	\end{flushright}
\end{minipage}}
 
% Zeilenabstand
\onehalfspacing	

%Beschriftung der Titelseite
\begin{center}

	\vspace*{4cm} %4 cm Vorspann
	\Large
	\textbf{Entwicklerhandbuch zur Losgrößen- und Ressourceneinsatzplanung bei Fließproduktion}\\ %Titel der Arbeit
	\large
	\textbf{Klassisches Losgrößenmodell, Mehrproduktproduktion}\\ %Untertitel der Arbeit
		
	\vspace*{8cm} %8 cm Vorspann
	\normalsize
	\begin{center}
	Juni 2015\\
	\textbf{Arnold Christiane \\ Denzin Timo \\ Eichinger Tobias \\ Sonnleitner Daniel \\ Wagner Pilar} %Name des Autors
	
	\end{center}
\end{center}
\pagebreak

% ------------------------------------------------------------------------------
% Inhaltsverzeichnis
% ------------------------------------------------------------------------------
% Inhaltsverzeichnis
\singlespacing %Zeilenabsatnd reduzieren
\setcounter{section}{0}
\setcounter{page}{1}
\addcontentsline{toc}{section}{Inhaltsverzeichnis}%hinzufügen des Inhaltsverzeichnises selbst

\tableofcontents %Ausgabe des Inhaltsverzeichnisses
\pagebreak

% ------------------------------------------------------------------------------
% Setzen der Nummerierungen für Normaltext
% ------------------------------------------------------------------------------
\onehalfspacing %Zeilenabstand auf 1.5
\renewcommand{\thesection}{\arabic{section}} %Arabische Beschriftung für Absatznummern
\pagenumbering{arabic}  %Seitennummerrierung auf arabisch setzten
\setcounter{page}{1}	%Seitenzahl für Inhalt auf 1 setzten
\setcounter{section}{0}
% Kopfzeile mit aktuellem Hauptkapitel darstellen
\renewcommand{\sectionmark}[1]{\markright{#1}} %Section ausgeben
\renewcommand{\subsectionmark}[1]{}            %Subsection nicht ausgeben
\renewcommand{\subsubsectionmark}[1]{}         %Subsubsection nicht ausgeben
\rhead{\rightmark}                             %Ausgabe Rechtsbündig

%------------------------------------------------------------------------------
%	Einführung
%------------------------------------------------------------------------------
\section{Einführung}
Das vorliegende Dokument ist ein Handbuch für Entwickler des \textit{"\gls{PLSP}"}-Tools zur Losgrößen- und Ressourceneinsatzplanung bei Fließproduktion. 
\\
Die Software ist in Java 8 geschrieben und die Benutzeroberfläche wurde mittels JavaFX 8 realisiert.

\subsection{Aufbau des Dokuments}
Zu Beginn soll die Problemstellung erklärt und der Ablauf des hier verwendeten Verfahrens erläutert werden.
\\
Anschließend folgt eine kurze Beschreibung der Funktionen der Software.
\\
Unter dem Punkt Aufbau der Software sollen die einzelnen Komponenten beschrieben und deren Zusammenspiel erläutert werden.
\\
Im letzten Kapitel wird auf die Implementierung der einzelnen Komponenten eingegangen.
\subsection{Problemstellung}
Das vorliegende Programm, dient zur Lösungen von, Losgrößen- und Ressourceneinsatzplanung bei Fließproduktion. Dabei werden die sowohl die Produktionszyklen für ein einzelnes Produkt, als auch der gemeinsame Zyklus für mehrere Produkte berechnet.
\subsection{Beschreibung der Software}
Nach dem Start des Tools, müssen die Daten zur Berechnung eingegeben werden, diese können entweder manuell eingegeben werden, oder von einer Datei geladen werden. Momentan wird nur der Import von \gls{CSV}-Dateien unterstützt.
\\
Nachdem die Produktionszyklen berechnet wurden, werden in den verschiedenen Tabs, die Ergebnisse in Form von Tabellen und Diagrammen dargestellt. Bei einem Klick in eine Zeile der Tabellen, erscheint in der Erklärkomponente die Berechnung dieses Wertes.
\\
Detailliertere Information über die Funktionen der Software finden Sie im Benutzerhandbuch.


%------------------------------------------------------------------------------
%	Aufbau der Software
%------------------------------------------------------------------------------

\section{Aufbau der Software}
Im Folgenden wird die verwendete Technik und die Komponenten des Programms vorgestellt. Die Komponenten werden durch Packages realisiert, daher werden auch die Packagenamen verwendet. Im weiteren Verlauf werden die Begriffe Komponente und Package gleichbedeutend verwendet.

\subsection{Verwendete Techniken}
\paragraph{JavaFX 8}
Die Benutzeroberfläche der Software wurde in JavaFX 8 entwickelt. JavaFX vewendet das Model-View-Controller Prinzip. Bei diesem Aufbau enthält die View lediglich den grafischen Aufbau der Oberfläche. Dieser Aufbau wird ein einer speziellen \gls{XML} Dateien gespeichert, sogenannten FXML Dateien. Eine zugehörige Controller Klasse implementiert die zur View gehörige Logik in einer Java Klasse. Im Model werden die verwendeten Daten abgespeichert und verwaltet. Weder Model noch View enthalten Logik. Dies führt zu einer sauberen Schichtentrennung zwischen Benutzeroberfläche (View), Logik (Controller) und Datenhaltung (Model).

\paragraph{JFreeChart}
Zur Visualisierung der Losgrößen, wird das Java-Framework JFreeChart verwendet. Mit JFreeChart können verschiedene Diagramme erstellt werden, darunter auch die hier verwendeten Gantt-Diagramme. 

\paragraph{JUnit}
Um die implementierte Logik automatisiert testen zu können, werden JUnit - Tests verwendet. Dabei wird eine eigene Testklasse geschrieben, die feste Ein-  und Ausgabedaten enthält. Dann wird die zu testende Methode ausgeführt und die Rückgabewerte der Methode mit den erwarteten Ausgabedaten verglichen. Stimmen sie überein, war der Test erfolgreich.
\\
Dies ermöglicht vor allem bei größeren Methoden ein automatisiertes Testen ohne Zutun des Entwicklers.

\subsection{Komponenten}
\paragraph{algorithms}
Hier sind die Algorithmen zur Berechnung der Losgrößen implementiert. Es gibt jeweils eine Klasse zur Berechnung der klassischen Losgrößen und zur Berechnung der Mehrproduktlosgrößen. Beide Klassen implementierten ein Interface, das gemeinsame Methoden vorgibt. 
\\
Der Produktionsprozess wird in beiden Fällen gleich berechnet, deshalb wird hierfür die selbe Klasse verwendet.
\paragraph{formula}
Die Komponente formula enthält die Klassen zur Erzeugung der Formeln, die in der Erklärkomponente angezeigt werden sollen. Die hierbei erzeugten Strings enthalten die Formeln, in Form von \LaTeX Code. Diese werden dynamisch zur Laufzeit erzeugt und mit den passenden Daten gefüllt. Damit ist es möglich, zu jedem Ergebnis eine Formel mit eingesetzten Werten zu erzeugen.
\\
Für die Formeln der klassischen Losgrößen und der Mehrproduktlosgrößen ist jeweils eine eigene Klasse implementiert. Außerdem ist noch eine Klasse für die Formeln des Produktionsprozesses und für produktspezifische Formeln vorhanden.
\paragraph{messages}
Dieses Package verwaltet die angezeigten Texte und Beschriftungen der Software. Dabei wird ein Ressource Handler verwendet, der es ebenfalls ermöglicht, das Programm in eine andere Sprache zu übersetzen, ohne große Änderungen am Code vorzunehmen.
\paragraph{persistence}
Hier werden die die Dateizugriffe verwaltet. Hierfür wird für jeden Dateityp eine eigene Klasse erstellt, die von der Klasse AbstractFile erben. Somit müssen Funktionen wie das Laden der Datei nicht erneut implementiert werden.
\paragraph{util}
In dieser Komponente werden neben einigen Hilfsklassen, vor allem die Einstellungen des Tools verwaltet. Hierfür wurde eine Singleton Klasse Configuration geschaffen, die die Einstellungen der Software verwaltet.
\paragraph{zoom}
In der Komponente zoom sind die Klassen zum vergrößern und verkleinern der Benutzeroberfläche enthalten. Dies wird mittels Veränderungen in den \gls{CSS} Style realisiert.
\paragraph{error}
Hier werden die Exceptions und andere Fehlermeldungen verwaltet.
\paragraph{model}
Die Model - Klassen zur Datenerhaltung werden in diesem Package aufbewahrt. Dazu gehören die Abstraktionen von Produkt, Produktionsprozess und dem Ergebnis der Analyse.
\paragraph{test}
Die Komponente test enthält die Klassen zur Durchführung der JUnit - Test. 
\paragraph{view}
In dieser Komponenten befinden sich die Controller und FXML Dateien der grafischen Oberfläche. Diese ist dabei in ein RootLayout und die fünf Tabs unterteilt. Das RootLayout definiert hier den Rahmen für die Tabs, die Menüleiste und die Unterleiste der Applikation. Die Tabs sind in das RootLayout integriert, besitzen aber eigene FXML-Dateien und Controller-Klassen.

%------------------------------------------------------------------------------
%	Implementierung
%------------------------------------------------------------------------------

\section{Implementierung}
Dieses Kapitel beschreibt die Implementierung und die bei der Entwicklung der Komponente verwendeten Bibliotheken. Dabei wird auf vor allem auf die Algorithmen der Losgrößenberechnung eingegangen.
 
\subsection{Verwendete Bibliotheken}
Die verwendeten Bibliotheken sind im Ordner lib gespeichert, um einen pfadunabhängigen Import in den Java Build Path zu ermöglichen. Sie unterteilen sich in fünf Einsatzbereiche:
\begin{itemize}
	\item[CSV-Import:] Diese Bibliotheken bieten Klassen zum automatisierten Einlesen und Beschreiben von \gls{CSV}-Dateien an. Dies wird mittels einer Parser und Printer Klasse erreicht.
	\item[JavaFX:] Um JavaFX als grafische Oberfläche zu nutzten, sind einige zusätzliche Bibliotheken nötig. Außerdem sind auch Codebeispiele und vorgefertigte Dialoge enthalten.
	\item[JFreeChart:] Das einbinden dieser Bibliotheken ermöglicht es Daten mittels Diagrammen zu visualisieren. Dabei wird die Klasse JFreeChart mit Daten gefüllt und dann zur anzeige gebracht.
	\item[\LaTeX :] Die Formeln in der Erklärkomponente werden mittels \LaTeX Code erzeugt. Der zur Verfügung gestellten Klasse TeXFormula wird der \LaTeX Code übergeben und daraus ein Bild generiert.
	\item[Resource Handler:] Der Resource Handler ersetzt die Eclipse eigene Funktion Source $\rightarrow$ Externalize Strings. Diese dient zur Auslagerung der Strings im Java Code in eine eigene Properties Datei. Dies hat den Vorteil, dass der Code übersichtlicher wird und es außerdem die Texte in mehreren Sprachen vorliegen können, ohne in den Klassen etwas zu ändern.
\end{itemize}

Nachfolgende Abbildung zeigt die verwendeten Bibliotheken in der Ordnerstruktur.
\begin{figure}[H]
	\centering
	\includegraphics[]{Bilder/libraries} 
	\captionof{figure}[]{Verwendete Bibliotheken}
	\label{fig:osgi}
\end{figure}

\subsection{Algorithmen}
In diesem Programm werden zwei verschiedene Verfahren verwendet. Die klassische Losgrößenberechnung und die Mehrproduktlosgrößenberechnung. Zur Verdeutlichung wird Pseudocode verwendet.
\subsubsection{Klassiche Losgrößenberechnung}
Zu Beginn wird findet die Losgrößenberechnung für ein Produkt statt. Diese ist in fünf Schritte unterteilt:
\begin{enumerate}
	\item Berechne die Lose für jedes Produkt
	\item Berechne die Produktionszeit für jedes Produkt
	\item Berechne die Effizienz der Maschine
	\item Berechne den optimalen Produktionszyklus für jedes Produkt
	\item Berechne die Reichweite für die einzelnen Produkte
\end{enumerate}
Nachfolgend ist der Ablauf in Pseudocode dargestellt.
\begin{lstlisting}[caption= Beispielprogramm, label=lst:code]
classicLotScheduling(List<Product> products){

    Map<Integer, Double> tOptSingle = new HashMap<Integer, Double>();
    LotSchedulingResult result;

	// calculateBatchSize 1
	for (Product product : products) {
	    product.setQ(Math.sqrt((2 * product.getD() * product.getS())
		    / (product.getH() * (1 - (product.getD() / product.getP())))));
	}
	// calculateProductionTime 2
	for (Product product : products) {
	    product.setT(product.getQ() / product.getP());
	}
	// calculateEfficiencyOfMachine 3
	for (Product product : products) {
	    product.setRoh(product.getD() / product.getP());
	}
	// calculateOptProductionCycle 4
	for (Product product : products) {
	    tOptSingle.put(
		    product.getK(),
		    Math.sqrt((2 * product.getS() / (product.getH()
	}
	// calculateRange 5
	for (Product product : products) {
	    product.setR(product.getQ() / product.getD());
	}
	return new LotSchedulingResult(products, tOptSingle);
}
\end{lstlisting}

\subsubsection{Mehrproduktlosgrößen}
Im zweiten Durchlauf findet die Losgrößenberechnung für mehrere Produkt statt. Diese ist in sechs Schritte unterteilt:
\begin{enumerate}
	\item Berechne die Effizienz der Maschine
	\item Berechne den Produktionszyklus für die Produkte
	\item Berechne den minimalen Produktionszyklus
	\item Berechne den optimalen gemeinsamen Produktionszyklus
	\item Berechne die Produktionszeit für jedes Produkt
	\item Berechne die Reichweite für die einzelnen Produkte
\end{enumerate}
Nachfolgend ist der Ablauf in Pseudocode dargestellt.
\begin{lstlisting}[caption= Beispielprogramm, label=lst:code]
MoreProductLotScheduling(List<Product> products){
	
    LotSchedulingResult result;
    Double tOpt;
	Double tMin;

	// calculateEfficiencyOfMachine 1
	for (Product product : products) {
	    product.setRoh(product.getD() / product.getP());
	}

	// calculateOptProductionCycle 2
	double numerator = 0.0;
	double denominator = 0.0;

	for (Product product : products) {
	    numerator += product.getS();
	}

	numerator *= 2;

	for (Product product : products) {
	    denominator += (product.getH() * product.getD() * (1 - product
		    .getRoh()));
	}

	tOpt = Math.sqrt(numerator / denominator);

	// calculateMinProductionCycle 3
	double numerator = 0.0;
	double denominator = 0.0;

	for (Product product : products) {
	    numerator += product.getTau();
	}

	for (Product product : products) {
	    denominator += product.getRoh();
	}

	denominator = 1 - denominator;

	tMin = (numerator / denominator);

	if (tOpt < tMin) {
	    throw new MinimalProductionCycleError();
	}

	// calculateBatchSize 4
	for (Product product : products) {
	    product.setQ(product.getD() * tOpt);
	}

	// calculateProductionTime 5
	for (Product product : products) {
	    product.setT(product.getQ() / product.getP());
	}

	// calculateRange 6
	for (Product product : products) {
	    product.setR(product.getQ() / product.getD());
	}

	return new LotSchedulingResult(products, tOpt, tMin);
}
\end{lstlisting}
%-------------------------------------------------------------------------------------
% Verzeichnisse %-------------------------------------------------------------------------------------
%	\rhead{Verzeichnisse} %Kopftextbeschriftung
%
%	\stepcounter{section}
%	\phantomsection \label{Verzeichnisse}
%	\addcontentsline{toc}{section}{Verzeichnisse} %Ohne Nummer ins Inhaltsverzeichnis
%	\renewcommand{\thesection}{\Roman{verzeichnis}}
	
	% Abkürzungen
%	\stepcounter{verzeichnis}
	\section{Abkürzungsverzeichnis}
	\vspace{-6em} % Abstand analog der anderen Verzeichnisse reduzieren
%	\printnoidxglossary[type=\acronymtype,style=alttree,title=,toctitle=] %automatischen Titel und Gliederungsbeschriftung unterdrücken - sonst steht da Glossarie
	


\end{document}
